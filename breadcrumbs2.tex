\documentclass{memoir}

% Plan of attack:
% 1. Create a data structure to maintain all of the levels of the heirarchy
% 2. Use that data structure and update it at every page-out.
%    a. Print this to the console on a debug basis.
% ?. Create a few example page styles that use that stack
% 4. Profit.

\ExplSyntaxOn



% Puts the name of the weight of the node in #1 into #2
\cs_new:Nn \__bc_node_get_weight_name:N
  { __c_ \cs_to_str:N #1 __bc_node_weight_int }

% Sets the weight of the node in #1 to #2
\cs_new:Nn \__bc_node_set_weight:Nn
  {
    \int_const:Nn
      { \__bc_node_get_weight_name:N #1 }
      { #2 }
  }

% Gets the weight of node #1 as an integer
\cs_new:Nn \__bc_node_get_weight:N
  { \use:c { \__bc_node_get_weight_name:N #1 } }



% Puts the name of the name of the node in #1 into the input stream
\cs_new:Nn \__bc_node_get_name_name:N
  { __c_ \cs_to_str:N #1 __bc_node_name_tl }

% Sets the name of the node in #1 to #2
\cs_new:Nn \__bc_node_set_name:Nn
  {
    \tl_const:Nn
      { \__bc_node_get_name_name:N #1 }
      { #2 }
  }

% Puts the weight of node #1 into the input stream
\cs_new:Nn \__bc_node_get_name:N
  { \use:c { \__bc_node_get_name_name:N #1 } }

% This will contain the current stack that represents the breadcrumb trail
\seq_new:N \g_breadcrumbs_trail_seq

% Creates a new weighted node to, presumably, be put onto the stack.
% #1 is the return value
% #2 is a value to push
% #3 is an integer weight to give it
\cs_new:Nn \__bc_node_new:Nnn
  {
    
  }

\ExplSyntaxOff

\usepackage{xparse,expl3}

% Define the logical behavior


\usepackage{mwe}
\begin{document}
\chapter{Introduction}
\lipsum
\section[Structure]{Structure of the System}
\lipsum
\section[Comparison]{Comparison with Other Graphics Packages}
\lipsum
\section{Utility Packages}
\lipsum
\section[How to Read]{How to Read This Manual}
\lipsum
\section[Acknowledgments]{Authors and Acknowledgments}
\lipsum
\section[Help]{Getting Help}
\lipsum
\part{Tutorials and Guidelines}
\chapter[Karl's Students]{Tutorial: A Picture for Karl's Students}
\lipsum
\section[Problem]{Problem Statement}
\lipsum
\section[Setting Up]{Setting up the Environment}
\lipsum
\subsection[\LaTeX]{Setting up the Environment in \LaTeX}
\lipsum
\subsection[\TeX]{Setting up the Environment in Plain \TeX}
\lipsum
\subsection[Con\TeX t]{Setting up the Environment in Con\TeX t}
\lipsum
\section[Straight Paths]{Straight Path Construction}
\lipsum
\section[Curved Paths]{Curved Path Construction}
\lipsum
\section[Circle Paths]{Circle Path Construction}
\lipsum
\section[Rectangle Paths]{Rectangle Path Construction}
\lipsum
\end{document}

% Local Variables:
% mode: latex
% TeX-master: t
% TeX-PDF-mode: t
% End:
