% \iffalse meta-comment
% !TEX program  = pdfLaTeX
%<*internal>
\iffalse
%</internal>
%<*readme>
----------------------------------------------------------------
brdcrmbs --- breadcrumbs for LaTeX
E-mail: seallred@smcm.edu
Released under the LaTeX Project Public License v1.3c or later
See http://www.latex-project.org/lppl.txt
----------------------------------------------------------------

Breadcrumbs for LaTeX
%</readme>
%<*internal>
\fi
\def\nameofplainTeX{plain}
\ifx\fmtname\nameofplainTeX\else
  \expandafter\begingroup
\fi
%</internal>
%<*install>
\input docstrip.tex
\keepsilent
\askforoverwritefalse
\preamble
----------------------------------------------------------------
brdcrmbs --- Breadcrumbs for LaTeX
E-mail: seallred@smcm.edu
Released under the LaTeX Project Public License v1.3c or later
See http://www.latex-project.org/lppl.txt
----------------------------------------------------------------

\endpreamble
\postamble

Copyright (C) 2013 by Sean Allred <seallred@smcm.edu>

This work may be distributed and/or modified under the
conditions of the LaTeX Project Public License (LPPL), either
version 1.3c of this license or (at your option) any later
version.  The latest version of this license is in the file:

http://www.latex-project.org/lppl.txt

This work is "maintained" (as per LPPL maintenance status) by
You.

This work consists of the file  brdcrmbs.dtx
and the derived files           brdcrmbs.ins,
                                brdcrmbs.pdf and
                                brdcrmbs.sty.

\endpostamble
\usedir{tex/latex/brdcrmbs}
\generate{
  \file{\jobname.sty}{\from{\jobname.dtx}{package}}
}
%</install>
%<install>\endbatchfile
%<*internal>
\usedir{source/latex/brdcrmbs}
\generate{
  \file{\jobname.ins}{\from{\jobname.dtx}{install}}
}
\nopreamble\nopostamble
\usedir{doc/latex/brdcrmbs}
\generate{
  \file{README.txt}{\from{\jobname.dtx}{readme}}
}
\ifx\fmtname\nameofplainTeX
  \expandafter\endbatchfile
\else
  \expandafter\endgroup
\fi
%</internal>
%<*package>
\NeedsTeXFormat{LaTeX2e}
\ProvidesPackage{brdcrmbs}[2013/07/01 v0.1 Breadcrumbs for LaTeX]
%</package>
%<*driver>
\documentclass{ltxdoc}
\usepackage[T1]{fontenc}
\usepackage{lmodern}
\usepackage{\jobname}
\usepackage[numbered]{hypdoc}
\usepackage[colorlinks]{hyperref}
\EnableCrossrefs
\CodelineIndex
\RecordChanges
\begin{document}
  \DocInput{\jobname.dtx}
\end{document}
%</driver>
% \fi
% 
%\GetFileInfo{\jobname.sty}
%
%\title{^^A
%  \textsf{brdcrmbs} --- Breadcrumbs for \LaTeX\thanks{^^A
%    This file describes version \fileversion, last revised \filedate.
%    Many thanks to the users of \TeX.SX, without whom this package
%    would be just \href{http://tex.stackexchange.com/questions/122792}
%    {a \emph{really} good question}.
%  }^^A
%}
%\author{^^A
%  Sean Allred\thanks{E-mail: seallred@smcm.edu}^^A
%}
%\date{Released \filedate}
%
%\maketitle
%
%\changes{v1.0}{2013/10/01}{First public release}
%
% A breadcrumb trail (or simply breadcrumbs) is a popular navigational tool for programs and documents that have an intrinsic hierarchical structure.
% Drawing its name from the story of Hansel~and~Gretel, a trail of breadcrumbs essentially depicts your location in the hierarchy.
% Since nearly all \TeX-based documents are \emph{structured}, this package aims to \emph{keep track of} and \emph{summarize} this structure as the document progresses.
% This package is designed to keep meticulous track of the structure internally, displaying this data only how and when desired.
% Taking advantage of the seperation of content and data in \LaTeX 3, this package aims to be implementation-agnostic.
%
% \section{Usage}
%\DescribeMacro{\BreadcrumbSet}
% Some text about an example macro called \cs{BreadcrumbSet}, which might have an optional argument \oarg{arg1} and mandatory one \marg{options}.
% \section{Expansion}
% Since the data is maintained separately from any specific implementation, this package is easily customized to a whim.
% However, while the end-user interface is in the style of \LaTeXe, this level of customization requires knowledge of \LaTeX 3 syntax, specifically \textsl{l3seq}.
% \DescribeMacro{\bc\_describe\_output:Nn}
% Arg 1 is a name for the output, Arg 2 is handler that takes the sequence \cs{g\_breadcrumb\_stack} and builds output from it.
% \emph{While you have access to this variable, don't screw it up.}
% I have not yet convinced myself this is even a good idea -- I may use the \cs{l} prefix and copy it over to that before giving it to you, negating loss of data.
%
%\StopEventually{^^A
%  \PrintChanges
%  \PrintIndex
%}
%
%    \begin{macrocode}
%<*package>
%    \end{macrocode}
%    
%\begin{macro}{\examplemacro}
%\changes{v1.0}{2009/10/06}{Some change from the previous version}
%    \begin{macrocode}
\newcommand*\examplemacro[2][]{%
  Some code here, probably
}
%    \end{macrocode}
%\end{macro} 
%
%    \begin{macrocode}
%</package>
%    \end{macrocode}
%\Finale
